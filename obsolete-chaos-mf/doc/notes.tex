 
%----------------------------------------------------------------------------------------
%	PACKAGES AND DOCUMENT CONFIGURATIONS
%----------------------------------------------------------------------------------------

\documentclass{article}

\usepackage{hyperref}
\usepackage{listings}
%\usepackage{siunitx} % Provides the \SI{}{} and \si{} command for typesetting SI units
\usepackage{graphicx} % Required for the inclusion of images
\usepackage{natbib}   % Required to change bibliography style to APA
\usepackage{amsmath}  % Required for some math elements
\usepackage{graphicx}
\usepackage{xcolor}
\usepackage{listings}

\setlength\parindent{0pt} % Removes all indentation from paragraphs

\renewcommand{\labelenumi}{\alph{enumi}.} % Make numbering in the enumerate environment by letter rather than number (e.g. section 6)

% warning box
\definecolor{warningbackground}{RGB}{252,226,158}
%\usepackage{floatflt}
\newcommand{\alertwarningbox}[1]{
    \centering
    \colorbox{warningbackground}{\parbox{400pt} {
            \vskip 10pt
            \begin{floatingfigure}[l]{50pt}
                \includegraphics[scale=.05]{img/danger.pdf}
            \end{floatingfigure}
            #1
            \vskip 10pt
        }
    }
}

\lstset{
language=C++,
basicstyle=\ttfamily,
keywordstyle=\color{blue}\ttfamily,
stringstyle=\color{red}\ttfamily,
commentstyle=\color{green}\ttfamily,
frame = single,
morecomment=[l][\color{magenta}]{\#}
}
%----------------------------------------------------------------------------------------
%	DOCUMENT INFORMATION
%----------------------------------------------------------------------------------------

\title{The ngpt Software Package \\ Reference \& User Guide \\ Dionysos Satellite Observatory, NTUA} % Title

\author{Xanthos \textsc{Papanikolaou} \and Demitris \textsc{Anastasiou}} % Author name

\date{\today} % Date for the report

\begin{document}

\maketitle % Insert the title, author and date

\begin{center}
\begin{tabular}{l r}
First Revision: & June 06, 2015 \\
Library Uri:    & \url{https://github.com/xanthospap/chaos-ngpt} \\
Version:        & v0.1-0
\end{tabular}
\end{center}

\begin{abstract}
{\small 
This document describes the ngpt Software package developed at Dionysos Satellite Observatory (DSO), of National Technical University
of Athens (NTUA).
}
\end{abstract}

\section{Satellite Systems}

\subsection{General}
The Satellite Systems included in the package, are all GNSS for which data can be extracted. Hence, 
the basic reference for GNSS is the RINEX format. Different RINEX versions can handle various
combinations of GNSS. As of version 3.02, 6 GNSS are considered.
\begin{center}
\begin{tabular}{l l l}
\hline
GNSS     & Identifier &  Rinex Versions\\
\hline
Gps      & 'G'                &  all\\
Glonass  & 'R'                &  from v2.00\\
Galileo  & 'E'                &  from v2.11\\
Sbas     & 'S'                &  from v2.10\\
BeiDou   & 'B' (in v3.01 'C') &  from v3.02 (in v3.01 as Compass)\\
QZSS     & 'J'                &  from v3.02\\
\end{tabular}
\end{center}
Satellite Systems have various attributes, that affect the observed quantities. Each GNSS,
has:
\begin{itemize}

\item an \textbf{identifier} (e.g. GPS has 'G'). These are explicitely mentioned in each RINEX version.
This identifier can be easil represented using a \texttt{char}.

\alertwarningbox{COMPASS is a problem !! Must be careful when it is encountered in RINEX v3.01}

\item a list of valid \textbf{(nominal) frequencies}. GPS has (now) 3 valid frequencies, L1, L2,
and L5. These should be represented by something like a \texttt{dictionary}, holding pairs of
indexes and values, i.e. carrier frequency in MHz (.e.g. for GPS { (1,1575.42), (2,1227.60), (5,1176.45) }).

\alertwarningbox{These are nominal frequencies and not actual, because GLONASS frequencies
are satellite-dependent, e.g. for carrier 1, the frequency is 1602.0 + k * 9/16,
where 1602.0 is the nominal frequency for the 1st carrier band.}
\end{itemize}

\subsection{Source Code}
The satellite systems could efficiently be represented by an enumerator. But:

\alertwarningbox{what about the MIXED or UNKNOWN satellite system ? Should they be
inluded?}

\begin{lstlisting}
enum class SATELLITE_SYSTEM : char {
    GPS,
    GLONASS,
    SBAS,
    GALILEO,
    BDS,
    QZSS,
    MIXED /// Should this be used/declared ??
};
\end{lstlisting}

The attributes for each satellite system, could be defined in a special \texttt{traits} class.
It would be nice if all attributes could be accesible at compile time.
\begin{lstlisting}
/// Traits for Satellite Systems
template<SATELLITE_SYSTEM S>
struct SatelliteSystemTraits
{ };

/// Specialize traits for Satellite System Gps
template<>
struct SatelliteSystemTraits<SATELLITE_SYSTEM::GPS>
{
    /// Identifier
    static constexpr char identifier { 'G' };

    /// Return the nominal frequency, given a band
    static double nominal_frequency(short int i)
    {
        switch ( i ) {
            case (1) : return 1575.42e0;
            case (2) : return 1227.60e0;
            case (5) : return 1176.45e0;
            default  : throw std::runtime_error
            ("ngpt::SatelliteSystemTraits<SATELLITE_SYSTEM::GPS>::"
             "nominal_frequency -> Invalid Band !!");
        }
    }
};
\end{lstlisting}

Or, if we want to specialize per frequency (per satellite system): 
\begin{lstlisting}
/// Traits for Satellite Systems
template<SATELLITE_SYSTEM S>
struct SatelliteSystemTraits
{ };

/// Specialize traits for Satellite System Gps
template<>
struct SatelliteSystemTraits<SATELLITE_SYSTEM::GPS>
{
    /// Identifier
    static constexpr char identifier { 'G' };
    
    template<short int F>
    class NominalFrequency {
        /// Force failed compilation.
        static_assert(sizeof(F) != sizeof(F), "Unspecialized Frequency !!"); 
    };
    template<>
    class NominalFrequency<1> {
        static constexpr double value { 1575.42e0 };
    };
    
    /// Return the nominal frequency, given a band
    static double nominal_frequency(short int i)
    {
        switch ( i ) {
            case (1) : return NominalFrequency<1>::value;
            case (2) : return NominalFrequency<2>::value;
            case (5) : return NominalFrequency<5>::value;
            default  : throw std::runtime_error
            ("ngpt::SatelliteSystemTraits<SATELLITE_SYSTEM::GPS>::"
             "nominal_frequency -> Invalid Band !!");
        }
    }
};
\end{lstlisting}

\subsection{Pragmatic View}
Most of the times (e.g. whre reading from data files), we won't know what satellite system
we will encounter. Templates in these cases are not a big help.

What we will definitely need, is function(s) to translate identifiers to satellite systems,
e.g. 

\begin{lstlisting}
SATELLITE_SYSTEM charToSatelliteSystem(char c)
{
    switch ( c ) {
        case (SatelliteSystemTraits<SATELLITE_SYSTEM::GPS>::identifier) :
            return SATELLITE_SYSTEM::GPS;
    /* ... more cases ... */
    }
}
\end{lstlisting}

Also, functions to check if a frequency (index) is valid for a satellite system, and
return the value (MHz), e.g.

\begin{lstlisting}
bool SatelliteSystemHasFrequency(short int f,SATELLITE_SYSTEM s)
{
  switch ( s ) {
    case (SATELLITE_SYSTEM::GPS) :
      switch ( f ) {
        case (1) :
          return SatelliteSystemTraits<SATELLITE_SYSTEM::GPS>::NominalFrequency<1>::value;
        /* ...more bands... */
      /* ...more satellite systems */
    }
  }
}
\end{lstlisting}

Or maybe this is better ...
\begin{lstlisting}
/// Specialize traits for Satellite System Gps
template<>
struct SatelliteSystemTraits<SATELLITE_SYSTEM::GPS>
{
    /* ...various declerations... */
    
    static std::unordered_map<short int, double> frequencies
    {
        { 1, 1575.42e0 },
        { 2, 1227.60e0 },
        { 5, 1176.45e0 },
    };
    
    static bool frequency_value(short int f,double &val)
    {
        auto v = frequencies.find(f);
        if (v == frequencies.end() ) {
            return false;
        } else {
            val = v.second;
        }
        return true;
    }
    
    /* ...various declerations... */
}
\end{lstlisting}

Keep in mind that forming observables, satellite systems and searching/forming frequencies
are going to happen only seldom. Better to be clear and easily extendable.

\section{Frequencies}

Each satellite system uses certain frequency bands to emmit the signal.
The bands are indexed with (unsigned short) indexes. The indexes correspond to
frequency values. For example, for GPS band(1) = 1575.42e0MHz, band(2) = 1227.60e0MHz
and band(5) = 1176.45e0MHz.

\alertwarningbox{Again, note that GLONASS frequencies
are satellite-dependent, e.g. for band(1), the frequency is 1602.0 + k * 9/16,
where 1602.0 is the nominal frequency for the 1st carrier band.}

Frequencies are easily represented by a \texttt{short int} or \texttt{short unsigned int}.

Frequencies are satellite-system-dependent, so it would make sense to include them
in the satallite system traits. E.g. as an assosiative array.

\begin{lstlisting}
/// Specialize traits for Satellite System Gps
template<>
struct SatelliteSystemTraits<SATELLITE_SYSTEM::GPS>
{
    /* ...various declerations... */
    
    static std::unordered_map<short int, double> frequencies
    {
        { 1, 1575.42e0 },
        { 2, 1227.60e0 },
        { 5, 1176.45e0 },
    };
    
    /* ...various declerations... */
}
\end{lstlisting}

Or, a more compile-time, \texttt{constexpr} way (see \url{http://stackoverflow.com/questions/8026906/constexpr-and-initialization})

\begin{lstlisting}
/// Specialize traits for Satellite System Gps
template<>
struct SatelliteSystemTraits<SATELLITE_SYSTEM::GPS>
{
    /* ...various declerations... */
    
    struct freq_pair {
        short int idx;
        double val;
    };

    constexpr freq_pair freq_map[] = {
        { 1, 1575.42e0 },
        { 2, 1227.60e0 },
        { 5, 1176.45e0 }
    };
    
    constexpr bool match(short int i,short int j) {
        

    /* ...various declerations... */
}
\end{lstlisting}

\section{Observation Types}


\section{Attributes}

%----------------------------------------------------------------------------------------
%	BIBLIOGRAPHY
%----------------------------------------------------------------------------------------

\bibliographystyle{apalike}

\bibliography{sample}

%----------------------------------------------------------------------------------------


\end{document}
