%----------------------------------------------------------------------------------------
%	PACKAGES AND OTHER DOCUMENT CONFIGURATIONS
%----------------------------------------------------------------------------------------

\documentclass[9pt]{extarticle} % Default font size and paper size

\usepackage[top=2cm, bottom=1.5cm, left=1.7cm, right=1.7cm]{geometry}
\usepackage{marvosym}
\usepackage{fontspec} %for loading fonts
\usepackage{xunicode,xltxtra,url,parskip} %other packages for formatting
\RequirePackage{color,graphicx}
\usepackage[usenames,dvipsnames]{xcolor}
\usepackage{longtable}
\usepackage{titlesec} %custom \section
%Setup hyperref package, and colours for links
\usepackage{hyperref}
\usepackage{fancyhdr}
\usepackage{lastpage}
\usepackage{fontawesome5}
%
%-------------------------------------------------------%
% SET FONT
% Latin Modern Typewriter
%\usepackage{lmodern}
%\renewcommand*\familydefault{\ttdefault} %% Only if the base font of the document is to be typewriter style
%\usepackage[T1]{fontenc}
%-------------------------------------------------------%

%\setmainfont[SmallCapsFont = Fontin SmallCaps]{Fontin} % Main document font
\pagestyle{fancy}
\definecolor{linkcolour}{rgb}{0,0.2,0.6}
\hypersetup{colorlinks,breaklinks,urlcolor=linkcolour,linkcolor=linkcolour}
\defaultfontfeatures{Mapping=tex-text}

\titleformat{\section}{\Large\scshape\raggedright}{}{0em}{}[\titlerule]
\titlespacing{\section}{0pt}{2pt}{2pt}
\titleformat{\subsection}{\large\scshape\raggedleft}{}{0em}{}[\titleline{\color{blue}\titlerule[0.5pt]}]
\titlespacing{\subsection}{0pt}{1.5pt}{1.5pt}

\fancyhf{}\renewcommand{\headrulewidth}{0pt}
\lfoot{\footnotesize\textit{X. Papanikolaou}}
\cfoot{\footnotesize\textit{Curriculum Vitae}}
\rfoot{\thepage \footnotesize\textit{/} \pageref{LastPage}}
\renewcommand{\footrulewidth}{0.4pt}

\begin{document}

\pagestyle{empty} % Removes page numbering
\font\fb=''[cmr10]'' % Change the font of the \LaTeX command under the skills section

%----------------------------------------------------------------------------------------
%	NAME AND CONTACT INFORMATION
%----------------------------------------------------------------------------------------
\par{\centering{\Huge Xanthos \textsc{Papanikolaou}}\bigskip\par} % Your name
%
\centering{\small \faLinkedin{} \href{https://www.linkedin.com/in/xanthos-papanikolaou-3125a0147/}{Xanthos Papanikolaou} \faGithub{} \href{https://github.com/xanthospap}{xanthospap} \faBitbucket{} \href{https://bitbucket.org/\%7B0e50e3f1-d363-4bc7-a391-a5d492995de3\%7D/}{xanthos papanikolaou} \faResearchgate{} \href{https://www.researchgate.net/profile/Xanthos\_Papanikolaou}{Xanthos Papanikolaou}}\\
%
\medskip
%
\centering{\textsc{\href{http://www.ntua.gr/}{National Technical University of Athens}} \\
\href{http://www.survey.ntua.gr/}{School of Rural \& Surveying Engineering}, Department of Topography\\
Dionysos Satellite Observatory, Higher Geodesy Laboratory \\
Heroon Polytechniou 9, 15780 Zografos, Athens | \faPhone{} +30 210772592 | \faMobile*{} +30 694 270 6646 | \faInbox{} \href{mailto:xanthos@mail.ntua.gr}{xanthos@mail.ntua.gr}}
\medskip

\section{Summary} \par
Born in Volos, Greece on the 3\textsuperscript{rd} of September 1984.
I am a Rural \& Surveying Engineer currently working on PhD dissertation. Since 2010 i have been working on research 
projects involved mainly in GPS/GNSS data acquisition, management and processing, collaborating with various 
institutions such as the \href{http://www.ntua.gr/}{National Technical University of Athens}, 
\href{http://www.noa.gr/index.php?lang=en}{National Observatory of Athens}, 
\href{http://www.teiath.gr/?lang=en}{Athens University of Applied Sciences}, 
\href{https://web.mit.edu/}{Massachusetts Institute of Technology} and the 
\href{http://www.ox.ac.uk/}{University of Oxford} and have contributed in a number of studies. My area of expertise 
is Satellite Geodesy and Navigation and my research interests include big data management and acquisition, GNSS 
data processing, time-series analysis and computational mathematics. I also have teaching assistance in the 
fields of Satellite Geodesy and Theory of Errors and Adjustment. I have a strong programming background and
solid knowledge of English and German.
\medskip

%----------------------------------------------------------------------------------------
%	EDUCATION
%----------------------------------------------------------------------------------------
\section{Education}
\begin{tabular}{rp{13cm}}
%
  \textsc{2010-current} &\textbf{Ph.D. Candidate at School of Rural \& Surveying Engineering, N.T.U.A.}\\
   & Field: \textit{Crustal Deformation Studies}\\
   %& \small Επιβλέπων: Καθ. Χριστιάνα \textsc{Μητσακάκη}\\
%
  \textsc{June 2010} & \textbf{IAG School on References Frames}\\
    & International Association of Geodesy \small{(IAG)}, Aegean University, Mytilene, Lesvos, Greece June 7-12, 2010\\
%
  \textsc{2002-2009} & \textbf{School of Rural \& Surveying Engineering, N.T.U.A.}\\
  & Diploma Thesis: 
  \textit{Least square solutions of extensive GPS networks for tectonic monitoring-Case study of Euboea, Greece (In Greek)}\\
%
\end{tabular}
\medskip

%----------------------------------------------------------------------------------------
%	LANGUAGES
%----------------------------------------------------------------------------------------
\section{Languages}
\begin{tabular}{rp{13cm}}
%
\textsc{English} & Certificate of Proficiency in English, University of Michigan \\
\textsc{German} & Zertifikat Grundstufe, Goethe Institut\\
%
\end{tabular}
\medskip

%----------------------------------------------------------------------------------------
%	TEACHING
%----------------------------------------------------------------------------------------
\section{Teaching \& Academic Experience}
%
Teaching assistance in the following undergraduate courses at School of Rural \& Surveying Engineering, N.T.U.A. :
%
\begin{longtable}{rl}
%
\textsc{2010-2019} & 
    Geodesy IV (Satellite Geodesy, 6\textsuperscript{th} semester)\\
  & Geodesy V (Higher Geodesy, 5\textsuperscript{th} semester)\\
  & Field Work for Higher and Satellite Geodesy (8\textsuperscript{th} semester)\\
  & Field Work for Geodesy (4\textsuperscript{th} semester)\\
  & Applications of Higher and Satellite Geodesy (9\textsuperscript{th} semester)\\
&\\
%
\textsc{2010-2012} & 
  Theory of Errors and Adjustment I (5\textsuperscript{th} semester)\\
  & Theory of Errors and Adjustment II (8\textsuperscript{th} semester)\\
&\\
%
\textsc{2010-current} & 
  Participation in Diploma Thesis supervision (over 7 thesis)\\
&\\
%
\end{longtable}
\medskip

%----------------------------------------------------------------------------------------
%	COMPUTER SKILLS
%----------------------------------------------------------------------------------------
\section{Computer Skills}
%
\begin{longtable}{r|p{13cm}}
%
\multicolumn{2}{c}{} \\
  \textsc{Operating Systems} 
  & GNU/Linux (user/administrator)\\
  & UNIX/\href{https://www.freebsd.org/}{FreeBSD} (user/administrator)\\
  & Windows\textregistered (XP, 7, 10, Windows Server) \\
%
\multicolumn{2}{c}{} \\
  \textsc{Programing Languages}
  & solid knowledge and experience in C, C++, Python, shell scripting (bash) and Java\\
  & very good knowledge of MATLAB\textregistered, \href{https://www.gnu.org/software/octave/}{GNU/Octave}\\
  & some experience with perl and FORTRAN\\
%
\multicolumn{2}{c}{} \\
  \textsc{Databases}
  & MySQL\textregistered, SQLite\textregistered \\
%
\multicolumn{2}{c}{} \\
  \textsc{Web Design \& FrameWorks}
  & experience with htlm, css\\
  & \href{https://getbootstrap.com/}{bootstrap}, 
    \href{https://www.djangoproject.com/}{django}, 
    \href{https://d3js.org/}{d3.js}\\
%
\multicolumn{2}{c}{} \\
  \textsc{Software \& Tools}
   & \href{http://www.qgis.org/en/site/}{QGIS}, 
     \href{http://gmt.soest.hawaii.edu/}{Generic Mapping Tools},
     \href{http://www.latex-project.org/}{\LaTeX},
     \href{http://git-scm.com/}{git}\\
  & \href{https://www.gnu.org/home.en.html}{GNU} Development Tools (\href{https://www.sourceware.org/autobook/autobook/autobook_toc.html}{autotools}, \href{https://gcc.gnu.org/}{GCC}), \href{http://www.vim.org/}{vim}, \href{http://llvm.org/}{LLVM/Clang},\\
  & \href{https://en.wikipedia.org/wiki/LAMP\_%28software_bundle%29}{LAMP} setup and configuration\\
%
\multicolumn{2}{c}{} \\
  \textsc{GNSS processing} & 
  \textsc{\href{http://www.bernese.unibe.ch/}{Bernese GNSS Software}}\\
    & various GNSS processing software suits,\\
    & \href{http://www.unavco.org/software/data-management/gsac/gsac.html}{GSAC}, \href{http://www.unavco.org/software/data-processing/teqc/teqc.html}{teqc}
%
\end{longtable}
%
\medskip

%----------------------------------------------------------------------------------------
% MEMBERSHIP OF SCIENTIFIC ORGANIZATIONS
%----------------------------------------------------------------------------------------
\section{Member of}
\begin{longtable}{rp{13cm}}
\textsc{2009 - current} & Technical Chamber of Greece \href{http://web.tee.gr/}{(TCG)}\\
\textsc{2009 - current} & Hellenic Association of Rural \& Surveying Engineers \href{http://www.psdatm.gr/}{(HARSE)}\\
\end{longtable}
\medskip

%----------------------------------------------------------------------------------------
%	WORK EXPERIENCE 
%----------------------------------------------------------------------------------------
%
\section{Working Experience}
%
\begin{longtable}{r|p{13cm}}
%
% ---------------- %
\multicolumn{2}{c}{} \\
\textsc{2010-current} & freelance Surveying Engineer\\ 
%  & - τοπογραφικές αποτυπώσεις, εργασίες πεδίου και γραφείου \\
%  & - σχεδιασμό και επίλυση μετρήσεων στατικού και κινηματικού εντοπισμού GPS\\
%  & - έλεγχο και συντήρηση μόνιμων γεωδαιτικών δικτύων GPS\\
%  & - παροχή επικουρικού διδακτικού έργου στην σχολή ΑΤΜ του Ε.Μ.Π. \\
%
\multicolumn{2}{c}{} \\
\textsc{2018-2020} & \textbf{\textsc{External Researcher for the \href{https://www.epos-ip.org/}{HelPOS} Project, National Observatory of Athens}}\\
  & \textit{\small Hellenic Plate Observing System}\\
  & Support for \href{http://geodesy.gein.noa.gr:8000/nginfo/}{NOANET} GNSS network (including data streaming, acquisition, reception, editing, archiving and publishing). Development of NOANET \href{http://geodesy.gein.noa.gr:8000/nginfo/}{web page}\\
  & co-financed by Greece and the EU, NSRF 2014-2020\\
  \\
  & \textbf{\textsc{External Researcher for the \href{https://www.epos-ip.org/}{HelPOS} Project, National Technical University of Athens}}\\
  & \textit{\small Hellenic Plate Observing System}\\
  & Setup remote communication and data streaming for NTUA's GPS network; data processing and time-series analysis\\
  & co-financed by Greece and the EU, NSRF 2014-2020\\
%
\multicolumn{2}{c}{} \\
\textsc{2018-2019} & \textbf{\textsc{External Researcher for the \href{https://www.epos-ip.org/}{EPOS} Project, National Observatory of Athens}}\\
  & Worked on the design, development and testing of \href{https://dsolab.github.io/StrainTool/}{StrainTool}.\\
  & funding from the European Union’s Horizon 2020 research and innovation programme under grant agreement N 676564.\\
%
% ---------------- %
\multicolumn{2}{c}{} \\
\textsc{2016-2017} & \textbf{\textsc{\href{http://www.astro.noa.gr/gr/research/projects/esa/}{Disaster risk reduction using innovative data exploitation methods and space assets}}}\\
  & \textit{\small ESA EXPRESS PROCUREMENT (EXPRO+)/OPEN-COMPETITIVE AO/1-8130/14/F/MOS}\\
  & Worked on the design and development of a GNSS data repository (data, meta-data and product database) and web tools to access it (available \href{http://ddrgsac.noa-gsp.terradue.com/ddrgsac}{here}).\\
  & funded by \href{http://www.esa.int/ESA}{ESA}.\\
%
% ---------------- %
\multicolumn{2}{c}{} \\
\textsc{2015-current} & \textbf{\textsc{\href{https://www.epncb.oma.be/_densification/}{EUREF Densification} Project, National Technical University of Athens}}\\
  & \textit{\small A joint venture of agencies and institutions from European countries which operate and/or analyse the data from dense national GNSS ngetworks}\\
  & Responsible for NTUA's contribution to the Densification Project, including the design and implementation of NTUA's GNSS data collection, processing and analysis tools.\\
  & See e.g. \href{https://www.epncb.oma.be/_newseventslinks/workshops/EPNLACWS_2015/pdf/planning_DSO_contribution_to_EUREF_densification.pdf}{EUREF Analysis Centre Workshop 2015} and \href{http://epncb.oma.be/_newseventslinks/workshops/EPNLACWS_2017/pdf/05_EPN_densification/03_xpapanikolaou.pdf}{EUREF Analysis Centres Workshop 2017}\\
%
% ---------------- %
\multicolumn{2}{c}{} \\
\textsc{2014-2015} & \textbf{\textsc{\href{http://dionysos.survey.ntua.gr/SEISMO/index.html}{SEISMO}}}\\
  & \textit{South Aegean Geodynamic And Tsunami Monitoring Platform.}\\
  & Participation in database development for geodata (MySQL,\href{http://www.unavco.org/software/data-management/gsac/gsac.html}{GSAC});
    geodetic data processing and time series analysis (GNSS, tide-gauges); \\
  & funded by \href{http://www.espa.gr/en/Pages/Default.aspx}{NSRF} and \href{http://www.gsrt.gr/central.aspx?sId=119I428I1089I323I488743}{GSRT}.\\
%
% ---------------- %
\multicolumn{2}{c}{} \\
\textsc{2012-2017} & \textbf{\textsc{School of Rural \& Surveying Engineering, N.T.U.A.}}\\
  & Participation in the design and development of a website for the laboratories of Higher Geodesy and Dionysos Satellite Observatory\\
  & \url{http://dionysos.survey.ntua.gr}\\
%
% ---------------- %
\multicolumn{2}{c}{} \\
\textsc{2012-2015} & \textbf{\textsc{\href{http://excellence.minedu.gov.gr/thales/en/thalesprojects/380208}{Seismo Fear Hellarc}}}\\
  & \textit{Integrated understanding of SEISmicity, using innovative MethOdologies of Fracture mechanics along with EARthquake and non extensive statistical physics – Application to the geodynamic system of the HELLenic ARC.}\\
  & Participation in GPS data processing and time series analysis;\\
  & funded by \textquotedblleft THALES\textquotedblright{} project.\\
%
% ---------------- %
\multicolumn{2}{c}{} \\ 
\textsc{2012-2014} & \textsc{\href{http://excellence.minedu.gov.gr/thales/en/thalesprojects/379347}{LAVMO - Landslide Vulnerability Model}}\\
  & \textit{Landslides Risk Model Development using Remote Sensing techniques and interferometry.}\\
  & Participation in working package: Satellite Geodesy;\\
  & funded by \textquotedblleft THALES\textquotedblright{} project.\\
%
% ---------------- %
\multicolumn{2}{c}{} \\
\textsc{2012-2013} & \textbf{\textsc{\href{http://www.teiath.gr/?lang=en}{Athens University of Applied Sciences}}}\\
  & \textit{Evaluation of height information of the trigonometric and leveling network of Greece, as part of the consolidation of the European reference systems and control: Application in the areas of Attica and Thessaloniki.}\\
  & Participation in field work and GNSS data processing \\
  & funded by \textquotedblleft Archimides III - Strengthening research groups\textquotedblright{}.\\
%
% ---------------- %
\multicolumn{2}{c}{} \\
\textsc{2012-2013} & \textbf{\textsc{\href{http://www.treecomp.gr/}{Tree Company CO}}}\\
  & Processing and analysis of the continuous GNSS network URANUS, of Tree Company CO.\\
%
% ---------------- %
\multicolumn{2}{c}{} \\
\textsc{2011-2013} & \textbf{\textsc{School of Rural \& Surveying Engineering, N.T.U.A.}}\\
  & Part-time work at the \href{http://portal.survey.ntua.gr/main/geocenter/geocen-g.html}{Geoinformatics Center} of the School of Rural \& Surveying Engineering, N.T.U.A.\\
%
% ---------------- %
\multicolumn{2}{c}{} \\
\textsc{2011-2012} & \textbf{\textsc{\href{http://www.oasp.gr/}{Organization for Earthquake Planning and Protection}}}\\
  & GNSS processing and analysis in the island of Santorini as part of the activities of the Special Scientific Monitoring Committee for the Santorini volcano.\\
  & \url{http://dionysos.survey.ntua.gr/src/santorini_project.htm}\\
%
% ---------------- %
\multicolumn{2}{c}{}\\
%\textsc{2010-2012} & \textbf{\textsc{\href{http://www.landslides.gr/index.php?lang=el}{Περιφέρεια Πελοποννήσου ΕΟΧ}}}\\
\textsc{2010-2012} & \textbf{\textsc{\href{http://www.landslides.gr/index.php?lang=en}{County of Peloponnese, Greece}}}\\
  & \textit{Development of a monitoring system for instability slopes to prevent landslides and training of local public authorities in the Region of Peloponnese (EOX: EL0071).}\\
  & Participation in field work and GPS data processing;\\
  & funded by EOX and the Greek Public Investment Programme.\\
%
% ---------------- %
\multicolumn{2}{c}{}\\
\textsc{2010} & \textbf{\textsc{EGNOS performance in South latitudes, GEOTOPOS LC}}\\
  & Participation in field work.\\
%
\end{longtable}
\medskip

%----------------------------------------------------------------------------------------
%	PUBLICATIONS
%----------------------------------------------------------------------------------------
\section{Publications}
\medskip
%
\subsection*{Publications in Journals and Honory Volumes}
%
\begin{longtable}{r|p{14cm}}
%
\multicolumn{2}{c}{} \\
  \textsc{2021}
%
  & NOANET: A Continuously Operating GNSS Network for Solid-Earth Sciences in Greece
  \emph{Chousianitis K., Papanikolaou X., Drakatos G., Tselentis G-Akis}, Seismological Research Letters 2021, \href{https://doi.org/10.1785/0220200340}{10.1785/0220200340}\\
\multicolumn{2}{c}{} \\
  \textsc{2019}
%
  & Regional integration of long-term national dense GNSS network solutions
  \emph{Kenyeres A., Bellet J. G., Bruyninx C., Caporali A., de Doncker F., Droscak B., Duret A., Franke P., Georgiev I., Bingley R., Huisman L., Jivall L., Khoda O., Kollo K., Kurt A. I., Lahtinen S., Legrand J., Magyar B., Mesmaker D., Morozova K., Naigl J., Azdemir S., Papanikolaou X., Parseliunas E., Stangl G., Ryczywolski M., Tangen O. B., Valdes M., Zurutuza J., Weber M.}, GPS Solutions 23, 122 (2019), \href{https://doi.org/10.1007/s10291-019-0902-7}{doi.org/10.1007/s10291-019-0902-7}\\
%
\multicolumn{2}{c}{} \\
  \textsc{2017}
%
  & The 2008 Methoni earthquake sequence: the relationship between the earthquake cycle on the subduction interface and coastal uplift in SW Greece
  \emph{Howell A., Palamartchouk K., Papanikolaou X., Paradissis D., Raptakis C., Copley A., England P. and Jackson J.}, Geophysical Journal International, \href{https://doi.org/10.1093/gji/ggw462}{doi:10.1093/gji/ggw462}\\
%
\multicolumn{2}{c}{} \\
  \textsc{2015}
%
  & From quiescence to unrest : 20 years of satellite geodetic measurements at Santorini volcano, Greece,
  \emph{Parks MM, Moore JDP, Papanikolaou X, Biggs J, Mather TA, Pyle DM, Raptakis C, Paradissis D, Hooper A, Parsons B, and Nomikou, P}, Journal of Geophysical Research: Solid Earth, Vol. 120, No. 2, 01.01.2015, p. 1309-1328, \href{http://dx.doi.org/10.1002/2014JB011540}{http://dx.doi.org/10.1002/2014JB011540}.\\
%  
\multicolumn{2}{c}{} \\ 
  \textsc{2013}
%
  & Joint approach using satellite techniques for slope instability detection and monitoring,
  \emph{Drakatos G., Paradissis P., Anastasiou D., Elias P., Marinou A., Chousianitis K., Papanikolaou X., Zacharis V., Argyrakis P.,Papazissi K. and Makropoulos K.},
  International Journal of Remote Sensing, 34:6, 1879-1892, \href{http://www.tandfonline.com/doi/abs/10.1080/2150704X.2012.731089#.Uxni9meIaig}{doi:10.1080/2150704X.2012.731089}\\
%
\multicolumn{2}{c}{} \\ 
  \textsc{2012}
  & Mapping inflation at Santorini volcano, Greece, using GPS and InSAR,
  \emph{Papoutsis, I., X. Papanikolaou, M. Floyd, K. H. Ji, C. Kontoes, D. Paradissis, and V. Zacharis,}
  Geophys. Res. Lett., \href{http://www.agu.org/pubs/crossref/pip/2012GL054137.shtml}{doi:10.1029/2012GL054137}, 2012\\
%
  &\\
%
  & Evolution of Santorini Volcano dominated by episodic and rapid fluxes of melt from depth,
  \emph{M. M. Parks, J. Biggs, P. England, T. A. Mather, P. Nomikou, K. Palamartchouk, X. Papanikolaou, D. Paradissis, B. Parsons, D. M. Pyle, C. Raptakis and V. Zacharis},
  Nature Geoscience (Advance Online Publication), 2012, \href{http://www.nature.com/ngeo/journal/v5/n10/full/ngeo1562.html}{doi:10.1038/ngeo1562}, 2012\\
%
\multicolumn{2}{c}{} \\ 
  \textsc{2010}
  & Deformation studies in the Gulf of Patras, Western Greece,
  \emph{Papazissi K., Anastasiou D., Marinou A., Mitsakaki C., Papanikolaou X., Paradissis D.}, 
  Honorary Volume in honor of D.Arabelo, Professor of the Aristotle University of Thessaloniki, 2010\\
\end{longtable}
%

\subsection*{Presentations in Conferences}
%
\begin{longtable}{r|p{14cm}}
\multicolumn{2}{c}{} \\ 
  \textsc{2015}
%
  & Planning DSO contribution to EUREF densification project.
  \emph{X. Papanikolaou, D. Anastasiou, V. Zacharis, A. Marinou, E. Tita, and D. Paradissis},
  EUREF Analysis Centre Workshop, AIU Bern, Switzerland, October 14-15, 2015\\
%
  &\\
%
  & GNSS processing at DSO: recent activity and current status.
  \emph{D. Anastasiou, X. Papanikolaou, A. Marinou, V. Zacharis, S. Alatza, D. Paradissis},
  EUREF Analysis Centre Workshop, AIU Bern, Switzerland, October 14-15, 2015\\
%
  &\\
%
  & Routine Analysis of all available GNSS Stations in Greece: Processing Scheme and Dissemination of Products and Data.
  \emph{Papanikolaou X., Anastasiou D., Marinou A., Zacharis V., Paradissis D.},
  EGU General Assembly 2015, held 12-17 April, 2015 in Vienna, Austria.\\
%
  &\\
%
  & Geodetic and seismological analysis of the January 26, 2014 Cephalonia Island earthquake sequence.
  \emph{D. Anastasiou, G. Chouliaras, X. Papanikolaou, A. Marinou, V. Zacharis, J. Galanis, G. Drakatos, D. Paradissis},
  26\textsuperscript{th} IUGG General Assembly, June 22 - July 2, Prague, Czech Republic, 2015.\\
%  
  \textsc{2013}
  & The Santorini Inflation Episode, Monitored by InSAR and GPS,
  \emph{Papoutsis I., Papanikolaou X., Floyd M., Ji K.H., Kontoes C., Paradissis D., Anastasiou D., and Ganas A.},
  European Space Agency Living Planet Symposium, Sept. 2013, Edinburgh, UK.\\
%
  &\\
%
  & The Santorini inflation episode: from start to finish.,
  \emph{Papoutsis I., Papanikolaou X., Floyd M., Ji K.H., Kontoes C., Paradissis D., Anastasiou D., and Ganas A.},
  European Geosciences Union General Assembly, April 2013, Vienna, Austria.\\
%
  &\\
%
  & Shallow evolution of Santorini volcano constrained by InSAR and GPS measurements,
  \emph{Parks M., Biggs J., England P., Mather T., Nomikou P., Palamartchouk K., Papanikolaou X., Paradissis D., Parsons B., Pyle D., Raptakis C. and Zacharis V.},
  European Geosciences Union General Assembly, April 2013, Vienna, Austria.\\
%
\multicolumn{2}{c}{} \\
  \textsc{2012}
  & Monitoring slope instability using a combined GPS and InSAR approach,
  \emph{Drakatos G., Paradissis D., D. Anastasiou D., Elias P., Marinou A., Chousianitis K., Papanikolaou X., Zacharis V., Argyrakis P., Papazisi K. and Makropoulos K.},
  33\textsuperscript{rd} General Assembly, European Seismological Commission, Augoust 19-24, 2012, Moscow, Russia.\\
%
  &\\
%
  & Development of a monitoring platform for slope instability and sliding prevention : preliminary results,
  \emph{Drakatos G., Paradissis D., D. Anastasiou D., Elias P., Marinou A., Chousianitis K., Papanikolaou X., Zacharis V., Argyrakis P., Papazisi K. and MakropoulosK.},
  EGU General Assembly 2012, Geophysical Research Abstracts ,Vol. 14, EGU2012-2609.\\
%
\multicolumn{2}{c}{} \\
  \textsc{2011}
  & Estimating tectonic velocities in the Ionian region, 
  \emph{Marinou Α., Papanikolaou X., Paradissis D., Anastasiou D., Zacharis V., Tzavaras P., Papazissi K., and Mitsakaki C.},
  EGU General Assembly 2011 , Geophysical Research Abstracts ,Vol. 13, EGU2011-12464.\\
%
\multicolumn{2}{c}{} \\
  \textsc{2010}
  & Crustal Deformation in the Patras Gulf, Greece, from GPS Data Analysis,
  \emph{Anastasiou D., Marinou A., Mitsakaki C., Papazissi K., Papanikolaou X., Paradissis D.},
  15th General Assembly of Wegener, Istanbul, Turkey, 14 – 17 September 2010.\\
%
  &\\
%
  & An Automated Processing Scheme Designed for All Available Permanent GPS Stations in Greece,
  \emph{Papanikolaou X., Marinou A., Mitsakaki C., Papazissi K., Paradissis D., Zacharis V., Anastasiou D.},
  15th General Assembly of Wegener, Istanbul, Turkey, 14 – 17 September 2010.\\
%
  &\\
%
  & Deformation Studies in the Kaparelli Area, Central Greece,
  \emph{Marinou A., Ganas A., Papanikolaou X., Bosy J., Papazissi K., Anastasiou D., Paradissis D., Drakatos G., Kontny B., Cacon S., Papanikolaou M.},
  15th General Assembly of Wegener, Istanbul, Turkey, 14 – 17 September 2010.\\
%
  &\\
%
  & Evaluation of Santorini's recent activity via a comparison of space and terrestrial geodetic techniques; Preliminary results,
  \emph{Paradissis D., Zaharis V., Raptakis C., Marinou A., Papanikolaou X., Anastasiou D., Papazissi K., Parks M., Parsons B., England P., Pyle D.},
  1\textsuperscript{st} Meeting of Tectonic Geodesy, Athens, 25 January 2012.\\
%
\end{longtable}

% Footer
\vfill
\begin{center}
  \begin{footnotesize}
    Last Update: \today
%    \href{\footerlink}{\texttt{\footerlink}}
  \end{footnotesize}
\end{center}

\end{document}
