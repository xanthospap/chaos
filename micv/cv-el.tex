%----------------------------------------------------------------------------------------
%	PACKAGES AND OTHER DOCUMENT CONFIGURATIONS
%----------------------------------------------------------------------------------------

\documentclass[a4paper,10pt]{article} % Default font size and paper size

\usepackage[top=2cm, bottom=1.5cm, left=1.7cm, right=1.7cm]{geometry}
\usepackage{marvosym}
\usepackage{fontspec} %for loading fonts
\usepackage{xunicode,xltxtra,url,parskip} %other packages for formatting
\RequirePackage{color,graphicx}
\usepackage[usenames,dvipsnames]{xcolor}
\usepackage{longtable}
\usepackage[monogreek]{xgreek}
\usepackage{titlesec} %custom \section
%Setup hyperref package, and colours for links
\usepackage{hyperref}
\usepackage{fancyhdr}
\usepackage{lastpage}
\pagestyle{fancy}
\definecolor{linkcolour}{rgb}{0,0.2,0.6}
\hypersetup{colorlinks,breaklinks,urlcolor=linkcolour,linkcolor=linkcolour}
\defaultfontfeatures{Mapping=tex-text}
%http://www.greekfontsociety.gr/pages/en_typefaces20th.html
\setmainfont[Mapping=tex-text,Numbers=OldStyle]{GFS Artemisia}

\titleformat{\section}{\Large\scshape\raggedright}{}{0em}{}[\titlerule]
\titlespacing{\section}{0pt}{2pt}{2pt}
\titleformat{\subsection}{\large\scshape\raggedleft}{}{0em}{}[\titleline{\color{blue}\titlerule[0.5pt]}]
\titlespacing{\subsection}{0pt}{1.5pt}{1.5pt}

\fancyhf{}\renewcommand{\headrulewidth}{0pt}
\lfoot{\footnotesize\textit{Ξ. Παπανικολάου}}
\cfoot{\footnotesize\textit{Βιογραφικό Σημείωμα}}
\rfoot{\thepage \footnotesize\textit{/} \pageref{LastPage}}
\renewcommand{\footrulewidth}{0.4pt}

\begin{document}

%\pagestyle{empty} % Removes page numbering
%\font\fb=''[cmr10]'' % Change the font of the \LaTeX command under the skills section

%----------------------------------------------------------------------------------------
%	NAME AND CONTACT INFORMATION
%----------------------------------------------------------------------------------------

\par{\centering{\Huge \textsc{Ξάνθος} \textsc{Παπανικολάου}}\bigskip\par} % Your name

\section{Προσωπικά Στοιχεία}

\begin{tabular}{rl}
  \textsc{Τόπος και ημερομηνία γέννησης:} & Λάρισα | 03 Σεπτεμβρίου 1984\\
  \textsc{Στοιχεία Επικοινωνίας:} \\
  & \textsc{\href{http://www.ntua.gr/}{Εθνικό Μετσόβιο Πολυτεχνείο}} \\
  & \href{http://www.survey.ntua.gr/}{Σχολή Αγρονόμων και Τοπογράφων Μηχανικών}, Τομέας Τοπογραφίας\\
  & Κέντρο Δορυφόρων Διονύσου, Εργαστήριο Ανώτερης Γεωδαισίας \\
  \textsc{Διεύθυνση:} & Ηρώων Πολυτεχνείου 9, 15780 Ζωγράφος, Αθήνα \\
  \textsc{Τηλέφωνο:} & +30 210772592\\
  \textsc{email:} & \href{mailto:xanthos@mail.ntua.gr}{xanthos@mail.ntua.gr}
\end{tabular}
\medskip

%----------------------------------------------------------------------------------------
%	EDUCATION
%----------------------------------------------------------------------------------------
\section{Εκπαίδευση}
\begin{tabular}{rp{13cm}}

  \textsc{2010-σήμερα} &\textbf{Υποψήφιος Διδάκτορας στη Σχολή Αγρονόμων και Τοπογράφων Μηχανικών Ε.Μ.Π.}\\
   & Πεδίο: \textit{Επεξεργασία GNSS Δεδομένων για Μελέτη Παραμορφώσεων του Γήινου Φλοιού}\\
   & \small Επιβλέπων: Καθ. Χριστιάνα \textsc{Μητσακάκη}\\

  \textsc{Ιούνιος 2010} & \textbf{IAG School on References Frames}\\
    & International Association of Geodesy \small{(IAG)}, Aegean University, Mytilene, Lesvos, Greece June 7-12, 2010\\

  \textsc{2002-2009} & \textbf{Σχολή Αγρονόμων και Τοπογράφων Μηχανικών Ε.Μ.Π.}\\
  & Διπλωματική εργασία: 
  \textit{Διερεύνηση μεθόδων συνόρθωσης μεγάλων δικτύων GPS για την παρακολούθηση μετακινήσεων - εφαρμογή στην Εύβοια}\\

\end{tabular}
\medskip

%----------------------------------------------------------------------------------------
%	LANGUAGES
%----------------------------------------------------------------------------------------
\section{Γλώσσες}
\begin{tabular}{rp{13cm}}

\textsc{Αγγλικά} & Certificate of Proficiency in English, University of Michigan \\
\textsc{Γερμανικά} & Zertifikat Mittelstufe, Goethe Institut\\

\end{tabular}
\medskip

%----------------------------------------------------------------------------------------
%	TEACHING
%----------------------------------------------------------------------------------------
\section{Εκπαιδευτική \& Διδακτική Εμπειρία}

Στη Σχολή Αγρονόμων και Τοπογράφων Μηχανικών του Ε.Μ Πολυτεχνείου :

\begin{longtable}{rl}

\textsc{2010-σήμερα} & 
  Γεωδαισία IV (Δορυφορική Γεωδαισία, 6\textsuperscript{ο} εξάμηνο)\\
  & Γεωδαισία V (Ανώτερη Γεωδαισία, 5\textsuperscript{ο} Εξάμηνο)\\
  & Μεγάλες Ασκήσεις Ανώτερης και Δορυφορικής Γεωδαισίας (8\textsuperscript{ο} εξάμηνο)\\
  & Μεγάλες Ασκήσεις Γεωδαισίας (4\textsuperscript{ο} εξάμηνο)\\
  & Εφαρμογές Ανώτερης και Δορυφορικής Γεωδαισίας (9\textsuperscript{ο} εξάμηνο)\\
&\\

\textsc{2010-σήμερα} & 
  Συμμετοχή στην επίβλεψη διπλωματικών εργασιών (πάνω από 7)\\
&\\

\textsc{2010-2012} & 
  Θεωρία Σφαλμάτων και Συνορθώσεις Ι (5\textsuperscript{ο} Εξάμηνο)\\
  & Θεωρία Σφαλμάτων και Συνορθώσεις ΙΙ (8\textsuperscript{ο} Εξάμηνο)\\
&\\

\end{longtable}
\medskip

%----------------------------------------------------------------------------------------
%	COMPUTER SKILLS
%----------------------------------------------------------------------------------------
\section{Γνώσεις Η/Υ}

\begin{longtable}{r|p{13cm}}

\multicolumn{2}{c}{} \\
  \textsc{Λειτουργικά Συστήματα} 
  & GNU/Linux (user/administrator)\\
  & UNIX/\href{https://www.freebsd.org/}{FreeBSD} (user/administrator)\\
  & Windows XP\textregistered , Windows 7\textregistered \\

\multicolumn{2}{c}{} \\
  \textsc{Γλώσσες Προγραμματισμού}
  & άριστη γνώση C/C++\\
  & πολύ καλή γνώση MATLAB\textregistered, \href{https://www.gnu.org/software/octave/}{GNU/Octave}\\
  & πολύ καλή γνώση \href{http://www.python.org/}{Python}, 
    shell scripting\href{http://tiswww.case.edu/php/chet/bash/bashtop.html}{(bash)}, 
    htlm, javascript\\
  & πολύ καλή γνώση \href{http://www.mysql.com/}{MySQL}\texttrademark\\

\multicolumn{2}{c}{} \\
  \textsc{Λογισμικά Πακέτα} 
   & \href{http://www.qgis.org/en/site/}{QGIS}, 
   \href{http://gmt.soest.hawaii.edu/}{Generic Mapping Tools},
   \href{http://www.latex-project.org/}{\LaTeX},
   \href{http://git-scm.com/}{git}\\

\multicolumn{2}{c}{} \\
  \textsc{Επεξεργασία GNSS:} & 
  \textsc{\href{http://www.bernese.unibe.ch/}{Bernese GNSS Software}}\\
    & Γνώση πολλών εμπορικών πακέτων επεξεργασίας GNSS δεδομένων\\

\end{longtable}
\medskip

%----------------------------------------------------------------------------------------
% MEMBERSHIP OF SCIENTIFIC ORGANIZATIONS
%----------------------------------------------------------------------------------------
\section{Μέλος Επιστημονικών Φορέων}
\begin{longtable}{rp{13cm}}
\textsc{2009 - σήμερα} & Τεχνικό Επιμελητήριο Ελλάδας \href{http://web.tee.gr/}{(ΤΕΕ)}\\
\textsc{2009 - σήμερα} & Πανελλήνιος σύλλογος Διπλωματούχων Αγρονόμων και Τοπογράφων Μηχανικών \href{http://www.psdatm.gr/}{(ΠΣΔΑΤΜ)}\\
\end{longtable}
\medskip

%----------------------------------------------------------------------------------------
%	WORK EXPERIENCE 
%----------------------------------------------------------------------------------------

\section{Εργασιακή Εμπειρία}

\begin{longtable}{r|p{13cm}}

%\emph{Current} & 1\textsuperscript{st} year Analyst at \textsc{Lehman Brothers}, London \\

% ---------------- %
\multicolumn{2}{c}{} \\
\textsc{2010-σήμερα} & ως ελεύθερος επαγγελματίας έχω απασχοληθεί με :\\ 
  & - τοπογραφικές αποτυπώσεις, εργασίες πεδίου και γραφείου \\
  & - σχεδιασμό και επίλυση μετρήσεων στατικού και κινηματικού εντοπισμού GPS\\
  & - έλεγχο και συντήρηση μόνιμων γεωδαιτικών δικτύων GPS\\
  & - παροχή επικουρικού διδακτικού έργου στην σχολή ΑΤΜ του Ε.Μ.Π. \\

% ---------------- %
\multicolumn{2}{c}{} \\
\textsc{2014-2015} & \textbf{\textsc{SEISMO}}\\
  & \textit{South Aegean Geodynamic And Tsunami Monitoring Platform.}\\
  & Συμμετοχή στην ανάπτυξη γεωχωρικής βάσης δεδομένων· ανάπτυξη και διαχείρηση μέσω διαδικτυακής πλατφόρμας (MySQL,\href{http://www.unavco.org/software/data-management/gsac/gsac.html}{GSAC});
    επεξεργασία και ανάλυση δορυφορικών γεωδαιτικών δεδομένων· ανάλυση χρονοσειρών (GNSS, tide-gauges); \\
  & Χρηματοδοτήθηκε από \href{http://www.espa.gr/en/Pages/Default.aspx}{NSRF} και \href{http://www.gsrt.gr/central.aspx?sId=119I428I1089I323I488743}{GSRT}.\\

% ---------------- %
\multicolumn{2}{c}{} \\
\textsc{2012-σήμερα} & \textbf{\textsc{Seismo Fear Hellarc}}\\
  & \textit{Integrated understanding of SEISmicity, using innovative MethOdologies of Fracture mechanics along with EARthquake and non extensive statistical physics – Application to the geodynamic system of the HELLenic ARC.}\\
  & Συμμετοχή στην επίλυση δεδομένων μόνιμων σταθμών GPS και στην ανάλυση των χρονοσειρών.\\
  & Χρηματοδοτήθηκε στα πλαίσια του έργου "ΘΑΛΗΣ"\\

% ---------------- %
\multicolumn{2}{c}{} \\ 
\textsc{2012-σήμερα} & \textsc{\textbf{LAVMO - Landslide Vulnerability Model}}\\
  & \textit{Ανάπτυξη Μοντέλου Επικινδυνότητας Κατολισθήσεων με χρήση μεθόδων Τηλεπισκόπισης και Συμβολομετρίας.}\\
  & Συμμετοχή στο ΠΕ3: Δορυφορική γεωδαισία.\\
  & Χρηματοδοτήθηκε στα πλαίσια του έργου "ΘΑΛΗΣ"\\

% ---------------- %
\multicolumn{2}{c}{} \\
\textsc{2012-σήμερα} & \textbf{\textsc{ΤΕΙ Αθήνας}}\\
  & \textit{Αξιολόγηση υψομετρικής πληροφορίας χωροσταθμικού και τριγωνομετρικού δικτύου της Ελλάδας στο πλαίσιο της ενοποίησης των Ευρωπαϊκών συστημάτων αναφοράς και ελέγχου: Εφαρμογή στους Νομούς Αττικής και Θεσσαλονίκης.}\\
  & Συμμετοχή στις μετρήσεις υπαίθρου και στην επεξεργασία δεδομένων GPS\\
  & Αρχιμήδης ΙΙΙ – Ενίσχυση ερευνητικών ομάδων στο ΤΕΙ Αθήνας.\\

% ---------------- %
\multicolumn{2}{c}{} \\
\textsc{2012-σήμερα} & \textbf{\textsc{Σχολή Αγρονόμων \& Τοπογράφων Μηχ. Ε.Μ.Π.}}\\
  & Συμμετοχή στο σχεδιασμό και στην ανάπτυξη της ιστοσελίδας των εργαστηρίων Ανώτερης Γεωδαισίας – Κέντρου Δορυφόρων Διονύσου.\\
  & \url{http://dionysos.survey.ntua.gr}\\

% ---------------- %
\multicolumn{2}{c}{} \\
\textsc{2012-2013} & \textbf{\textsc{\href{http://www.treecomp.gr/}{Tree Company CO}}}\\
  & Επίλυση και επεξεργασία του δικτύου μόνιμων σταθμών GNSS URANUS της εταιρείας Tree Company CO.\\

% ---------------- %
\multicolumn{2}{c}{} \\
\textsc{2011-2013} & \textbf{\textsc{Σχολή Αγρονόμων \& Τοπογράφων Μηχ. Ε.Μ.Π.}}\\
  & Μερική απασχόληση στο \href{http://portal.survey.ntua.gr/main/geocenter/geocen-g.html}{Κέντρο Γεωπληροφορικής} της Σχολής Αγρονόμων και Τοπογράφων Μηχανικών του Ε.Μ.Π.\\

% ---------------- %
\multicolumn{2}{c}{} \\
\textsc{2011-2012} & \textbf{\textsc{\href{http://www.oasp.gr/}{Οργανισμός Αντισεισμικού Σχεδιασμού και Προστασίας}}}\\
  & Επεξεργασία μετρήσεων GPS και ανάλυση αποτελεσμάτων στην περιοχή της Σαντορίνης στο πλαίσιο των δραστηριοτήτων της Ειδικής Επιστημονικής Επιτροπής Παρακολούθησης του Ηφαιστείου της Σαντορίνης.\\

% ---------------- %
\multicolumn{2}{c}{}\\
\textsc{2010-2012} & \textbf{\textsc{\href{http://www.landslides.gr/index.php?lang=el}{Περιφέρεια Πελοποννήσου ΕΟΧ}}}\\
  & \textit{Ανάπτυξη συστήματος παρακολούθησης της αστάθειας κλιτύων για την πρόληψη κατολισθήσεων και εκπαίδευση των τοπικών δημοσίων αρχών στην Περιφέρεια Πελοποννήσου (Κωδικός ΕΟΧ: EL0071).}\\
  & Συμμετοχή στις εργασίες υπαίθρου για μετρήσεις GPS και επίλυση των μετρήσεων.\\
  & Χρηματοδοτήθηκε από τον ΕΟΧ και το Πρόγραμμα Δημοσίων Επενδύσεων.\\

% ---------------- %
\multicolumn{2}{c}{}\\
\textsc{2010} & \textbf{\textsc{EGNOS performance in South latitudes, ΓΕΩΤΟΠΟΣ Α.Ε.}}\\
  & Συμμετοχή στις εργασίες υπαίθρου για μετρήσεις GPS\\

\end{longtable}
\medskip

%----------------------------------------------------------------------------------------
%	PUBLICATIONS
%----------------------------------------------------------------------------------------
\section{Κατάλογος Δημοσιεύσεων}
\medskip

\subsection*{Άρθρα σε Περιοδικά και Τιμητικούς Τόμους}

%\begin{longtable}
\begin{longtable}{r|p{14cm}}
\multicolumn{2}{c}{} \\
  \textsc{2015}
  & From quiescence to unrest : 20 years of satellite geodetic measurements at Santorini volcano, Greece,
  \emph{Parks MM, Moore JDP, Papanikolaou X, Biggs J, Mather TA, Pyle DM, Raptakis C, Paradissis D, Hooper A, Parsons B, and Nomikou, P},
  Journal of Geophysical Research: Solid Earth, Vol. 120, No. 2, 01.01.2015, p. 1309-1328,
  \href{http://dx.doi.org/10.1002/2014JB011540}{http://dx.doi.org/10.1002/2014JB011540}.

\multicolumn{2}{c}{} \\
  \textsc{2013}
  & Ανάπτυξη συστήματος παρακολούθησης της αστάθειας κλιτύων για την πρόληψη των κατολισθήσεων: προκεταρκτικά αποτελέσματα.,
  \emph{Δρακάτος Γ., Παραδείσης Δ., Αναστασίου Δ., Ηλίας Π., Μαρίνου Α., Χουσιανίτης Κ., Παπανικολάου Ξ., Ζαχαρής Ε., Αργυράκης Π., Παπαζήση Κ., Μήλας Π.,}
  τιμητικός τόμος προς τιμήν του Δ. Βλάχου, Ομότιμου Καθηγητή του Αριστοτελείου Πανεπιστημίου Θεσσαλονίκης, 2013\\

  &\\

  & Joint approach using satellite techniques for slope instability detection and monitoring,
  \emph{Drakatos G., Paradissis P., Anastasiou D., Elias P., Marinou A., Chousianitis K., Papanikolaou X., Zacharis V., Argyrakis P.,Papazissi K. and Makropoulos K.},
  International Journal of Remote Sensing, 34:6, 1879-1892, \href{http://www.tandfonline.com/doi/abs/10.1080/2150704X.2012.731089#.Uxni9meIaig}{doi:10.1080/2150704X.2012.731089}\\

\multicolumn{2}{c}{} \\ 
  \textsc{2012}
  & Mapping inflation at Santorini volcano, Greece, using GPS and InSAR,
  \emph{Papoutsis, I., X. Papanikolaou, M. Floyd, K. H. Ji, C. Kontoes, D. Paradissis, and V. Zacharis,}
  Geophys. Res. Lett., \href{http://www.agu.org/pubs/crossref/pip/2012GL054137.shtml}{doi:10.1029/2012GL054137}, 2012\\

  &\\

  & Evolution of Santorini Volcano dominated by episodic and rapid fluxes of melt from depth,
  \emph{M. M. Parks, J. Biggs, P. England, T. A. Mather, P. Nomikou, K. Palamartchouk, X. Papanikolaou, D. Paradissis, B. Parsons, D. M. Pyle, C. Raptakis and V. Zacharis},
  Nature Geoscience (Advance Online Publication), 2012, \href{http://www.nature.com/ngeo/journal/v5/n10/full/ngeo1562.html}{doi:10.1038/ngeo1562}, 2012\\

\multicolumn{2}{c}{} \\ 
  \textsc{2010}
  & Deformation studies in the Gulf of Patras, Western Greece,
  \emph{Papazissi K., Anastasiou D., Marinou A., Mitsakaki C., Papanikolaou X., Paradissis D.}, 
  Honorary Volume in honor of D.Arabelo, Professor of the Aristotle University of Thessaloniki, 2010\\
\end{longtable}

\subsection*{Ανακοινώσεις σε Συνέδρια}

\begin{longtable}{r|p{14cm}}
\multicolumn{2}{c}{} \\ 
  \textsc{2013}
  & The Santorini Inflation Episode, Monitored by InSAR and GPS,
  \emph{Papoutsis I., Papanikolaou X., Floyd M., Ji K.H., Kontoes C., Paradissis D., Anastasiou D., and Ganas A.},
  European Space Agency Living Planet Symposium, Sept. 2013, Edinburgh, UK.\\

  &\\

  & The Santorini inflation episode: from start to finish.,
  \emph{Papoutsis I., Papanikolaou X., Floyd M., Ji K.H., Kontoes C., Paradissis D., Anastasiou D., and Ganas A.},
  European Geosciences Union General Assembly, April 2013, Vienna, Austria.\\

\multicolumn{2}{c}{} \\
  \textsc{2012}
  & Monitoring slope instability using a combined GPS and InSAR approach,
  \emph{Drakatos G., Paradissis D., D. Anastasiou D., Elias P., Marinou A., Chousianitis K., Papanikolaou X., Zacharis V., Argyrakis P., Papazisi K. and Makropoulos K.},
  33\textsuperscript{rd} General Assembly, European Seismological Commission, Augoust 19-24, 2012, Moscow, Russia.\\

  &\\

  & Development of a monitoring platform for slope instability and sliding prevention : preliminary results,
  \emph{Drakatos G., Paradissis D., D. Anastasiou D., Elias P., Marinou A., Chousianitis K., Papanikolaou X., Zacharis V., Argyrakis P., Papazisi K. and MakropoulosK.},
  EGU General Assembly 2012, Geophysical Research Abstracts ,Vol. 14, EGU2012-2609.\\

\multicolumn{2}{c}{} \\
  \textsc{2011}
  & Estimating of tectonic velocities in the Ionian region, 
  \emph{Marinou Α., Papanikolaou X., Paradissis D., Anastasiou D., Zacharis V., Tzavaras P., Papazissi K., and Mitsakaki C.},
  EGU General Assembly 2011 , Geophysical Research Abstracts ,Vol. 13, EGU2011-12464.\\

\multicolumn{2}{c}{} \\
  \textsc{2010}
  & Crustal Deformation in the Patras Gulf, Greece, from GPS Data Analysis,
  \emph{Anastasiou D., Marinou A., Mitsakaki C., Papazissi K., Papanikolaou X., Paradissis D.},
  15th General Assembly of Wegener, Istanbul, Turkey, 14 – 17 September 2010.\\

  &\\

  & An Automated Processing Scheme Designed for All Available Permanent GPS Stations in Greece,
  \emph{Papanikolaou X., Marinou A., Mitsakaki C., Papazissi K., Paradissis D., Zacharis V., Anastasiou D.},
  15th General Assembly of Wegener, Istanbul, Turkey, 14 – 17 September 2010.\\

  &\\

  & Deformation Studies in the Kaparelli Area, Central Greece,
  \emph{Marinou A., Ganas A., Papanikolaou X., Bosy J., Papazissi K., Anastasiou D., Paradissis D., Drakatos G., Kontny B., Cacon S., Papanikolaou M.},
  15th General Assembly of Wegener, Istanbul, Turkey, 14 – 17 September 2010.\\
\end{longtable}

\subsection*{Παρουσιάσεις σε Ημερίδες}

\begin{longtable}{r|p{14cm}}
\multicolumn{2}{c}{} \\
  \textsc{2012}
  & Evaluation of Santorini's recent activity via a comparison of space and terrestrial geodetic techniques; Preliminary results,
  \emph{Paradissis D., Zaharis V., Raptakis C., Marinou A., Papanikolaou X., Anastasiou D., Papazissi K., Parks M., Parsons B., England P., Pyle D.},
  1\textsuperscript{st} Meeting of Tectonic Geodesy, Athens, 25 January 2012.\\
\end{longtable}
\medskip

% Footer
\vfill
\begin{center}
  \begin{footnotesize}
    Τελευταία Ενημέρωση: \today
%    \href{\footerlink}{\texttt{\footerlink}}
  \end{footnotesize}
\end{center}

\end{document}
